\documentclass[10pt]{standalone}
\usepackage[utf8]{inputenc}

\usepackage{./articlethemer}
\usecolortheme{erdiagram}
\useinnertheme{erdiagram}

\usepackage{tikz}
\usetikzlibrary{shapes.multipart}
\usetikzlibrary{matrix}
\usetikzlibrary{calc}
\usetikzlibrary{fit}

% echo boolean
\def\echotrue{true}
\def\echofalse{false}
% echos directions
\def\echoeast{east}
\def\echowest{west}

\tikzset{%
    fill=pictureBack,
    every path/.append style={edgeColor},%
    every matrix/.append style={%
        matrix of nodes,%
        anchor=north%
    },%
    tables/.style={%
        every matrix/.append style={%
            fill=recordBack,%
            column sep=3pt,%
        },%
        row 1 column 3/.append style={%
            nodes={title}%
        }%
    },%
    layout/.style={%
        row sep=2in,%
        column sep=2in%
    },
    title/.style={%
        titleText,
        fill=titleBack,%
    },%
    key/.style={%
        keyText,%
        fill=#1,%
        font={\tiny\sffamily},%
        draw=keyBorder,%
        line width=0.5pt%
    },%
    key/.default={keyBack},%
    primary key/.style={key=primaryKeyBack},
    % the crow's feet
	zero/.pic={%
		\draw [thick, fill=pictureBack] (12pt, 0pt) circle (6pt);%
	},%
	one/.pic={%
		\draw [thick] (12pt, 6pt) -- (12pt, -6pt);%
	},%
	to one/.pic={%
		\draw [thick] (6pt, 6pt) -- (6pt, -6pt);%
	},%
	at least/.pic={%
		\draw [thick] (6pt, 0pt) -- (0pt, -6pt);%
		\draw [thick] (6pt, 0pt) -- (0pt, 0pt);%
		\draw [thick] (6pt, 0pt) -- (0pt, 6pt);%
	},%
}%

\begin{document}

\begin{tikzpicture}
    \matrix[layout] (layout) {
        {} & {} \\
        {} & {} \\
        {} & {} \\
    };
    \begin{scope}[tables]
        % Note: \vphantom{pl} is used to uniform the height
        % of each row
        \matrix (Account) at (layout-1-1) {
            \phantom{PK} & \phantom{FK} & Account%
            \vphantom{pl}\\
            |[primary key]|PK & {} & acid%
            \vphantom{pl}\\
            \hline
            {} & {} & password%
            \vphantom{pl}\\
        };
        \matrix (Profile) at (layout-1-2) {
            \phantom{PK} & \phantom{FK} & Profile%
            \vphantom{pl}\\
            |[primary key]|PK & {} & pfid%
            \vphantom{pl}\\
            \hline
            {} & |[key]|FK & acid%
            \vphantom{pl}\\
            {} & {} & pfLevel%
            \vphantom{pl}\\
            {} & {} & score%
            \vphantom{pl}\\
        };
        \matrix (Lesson) at (layout-2-2) {
            \phantom{PK} & \phantom{FK} & Lesson%
            \vphantom{pl}\\
            |[primary key]|PK & {} & lsid%
            \vphantom{pl}\\
            \hline
            {} & |[key]|FK & pfid%
            \vphantom{pl}\\
            {} & {} & isLevel%
            \vphantom{pl}\\
        };
        \matrix (Question) at (layout-2-1) {
            \phantom{PK} & \phantom{FK} & Question%
            \vphantom{pl}\\
            |[primary key]|PK & {} & qsid%
            \vphantom{pl}\\
            \hline
            {} & |[key]|FK & lsid%
            \vphantom{pl}\\
            {} & |[key]|FK & qsItid%
            \vphantom{pl}\\
        };
        \matrix (answers) at (layout-3-1) {
            \phantom{PK} & \phantom{FK} & answers%
            \vphantom{pl}\\
            |[primary key]|PK & {} & anid%
            \vphantom{pl}\\
            \hline
            {} & |[key]|FK & qsid%
            \vphantom{pl}\\
            {} & |[key]|FK & itid%
            \vphantom{pl}\\
            {} & {} & isCorrect%
            \vphantom{pl}\\
        };
        \matrix (Item) at (layout-3-2) {
            \phantom{PK} & \phantom{FK} & Item%
            \vphantom{pl}\\
            |[primary key]|PK & {} & itid%
            \vphantom{pl}\\
            \hline
            {} & {} & itName%
            \vphantom{pl}\\
            {} & {} & itSource%
            \vphantom{pl}\\
        };
    \end{scope}

    % add titles
    \foreach \table in {Account,Profile,Question,Lesson,answers,Item}{
        % corners of title
        \coordinate (\table-title-north west) at
            ($(\table-1-1.north-|\table.west)
                +(3pt,0)+(\pgflinewidth,0)$);
        \coordinate (\table-title-south east) at
            ($(\table-1-3.south-|\table.east)
                +(-3pt,4pt)+(-\pgflinewidth,0)$);
        % title background
        \node[
            fit=(\table-title-north west)%
                (\table-title-south east),%
            title,%
            draw=white,%
        ] (\table-title back) {\phantom{\table}};
        % title text
        \node[title,anchor=center] (\table-title text) at
        (\table-title back) {\table};
    };

    % edges
    % \from* = related to the table the edge comes from
    % \to* = related to the table the edge goes to
    % \*table = the actual table's name
    % \*row = the row of the key to link
    % \*side = the side of the table to leave/enter from
    % \*MUp = upper bound of multiplicty
    % \*MLo = lower bound of multiplicty
    % \doMidpoint, whether to calculate the overshoot for
    %       the midpoint used when the path has to go around a
    %       table
    \foreach \fromtable/\fromrow/\fromside/\fromMUp/\fromMLo/\totable/\torow/\toside/\toMUp/\toMLo/\doMidpoint in {
        Account/2/east/one/to one/%
            Profile/3/west/one/to one/%
            false,%
        Profile/2/east/one/to one/%
            Lesson/3/east/at least/zero/%
            false,%
        Lesson/2/west/one/to one/%
            Question/3/east/at least/one/%
            false,%
        Question/2/west/one/to one/%
            answers/3/west/at least/one/%
            false,%
        answers/4/east/at least/zero/%
            Item/2/west/one/to one/%
            false,%
        Question/4/east/one/to one/%
            Item/2/east/one/to one/%
            true%
    }{
        % \*portno = column from which to exit or to enter
        % \*sign = whether to add or subtract from the port
        % to get the terminal
        % \*footrot = rotation of the foot

        % set the side port numbers from according to
        % \fromside
        \ifx\fromside\echowest
            \def\fromportno{1}
            \def\fromsign{-}
            \def\fromfootrot{180}
        % \ifx\fromside\echowest
        \else\ifx\fromside\echoeast
            \def\fromportno{3}
            \def\fromsign{+}
            \def\fromfootrot{0}
        \fi\fi % \fromside\echoeast
        % set the side port numbers to according to \toside
        \ifx\toside\echowest
            \def\toportno{1}
            \def\tosign{-}
            \def\tofootrot{180}
        % \ifx\toside\echowest
        \else\ifx\toside\echoeast
            \def\toportno{3}
            \def\tosign{+}
            \def\tofootrot{0}
        \fi\fi % \toside\echoeast

        % edge of from port
        \coordinate (\fromtable-\fromrow-\fromside) at (\fromtable.\fromside|-\fromtable-\fromrow-\fromportno.\fromside);

        % edge of to port
        \coordinate (\totable-\torow-\toside) at (\totable.\toside|-\totable-\torow-\toportno.\toside);

        % terminal from
        \draw (\fromtable-\fromrow-\fromside)
        -- ++(\fromsign\terminalLength,0) coordinate (\fromtable-\fromrow-\fromside-terminal);
        % terminal to
        \draw (\totable-\torow-\toside) -- ++(\tosign\terminalLength,0) coordinate (\totable-\torow-\toside-terminal);
        % foot from
        \draw (\fromtable-\fromrow-\fromside)
        pic[rotate=\fromfootrot] {\fromMUp}
        pic[rotate=\fromfootrot] {\fromMLo};
        % foot to
        \draw (\totable-\torow-\toside)
        pic[rotate=\tofootrot] {\toMUp}
        pic[rotate=\tofootrot] {\toMLo};

        % mid point
        % set the side port numbers from according to
        % \fromside
        \ifx\doMidpoint\echotrue
            % if true, calculate the overshoot
            \coordinate (\totable-overshoot-\fromtable) at (\totable-\torow-\toside-terminal|-\fromtable-\fromrow-\fromside-terminal);
        % \ifx\doMidpoint\echotrue
        \else\ifx\doMidpoint\echofalse
            % if false, reuse the from terminal the overshoot
            \coordinate (\totable-overshoot-\fromtable) at (\totable-\torow-\toside-terminal);
        \fi\fi % \doMidpoint\echofalse

        % connect terminals
        \draw (\totable-\torow-\toside-terminal)
            -- ($(\totable-\torow-\toside-terminal)!0.5!(\totable-overshoot-\fromtable)$)
            -| (\fromtable-\fromrow-\fromside-terminal);
    }
\end{tikzpicture}

\end{document}
